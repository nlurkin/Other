%%This is a very basic article template.
%%There is just one section and two subsections.
\documentclass{article}

\usepackage{amsmath}
\usepackage{hyperref}
\usepackage{listings}
\usepackage{color}
\usepackage{subcaption}
\usepackage{graphicx}

\lstset{ %
  backgroundcolor=\color{white},   % choose the background color; you must add \usepackage{color} or \usepackage{xcolor}
  basicstyle=\tiny,        % the size of the fonts that are used for
  % the code
  breakatwhitespace=false,         % sets if automatic breaks should only happen at whitespace
  breaklines=true,                 % sets automatic line breaking
  %captionpos=b,                    % sets the caption-position to bottom
  commentstyle=\color{green},    % comment style
  %deletekeywords={...},            % if you want to delete keywords from the
  % given language
  %escapeinside={\%*}{*)},          % if you want to add LaTeX within your code
  %extendedchars=true,              % lets you use non-ASCII characters; for
  % 8-bits encodings only, does not work with UTF-8
  frame=L,                    % adds a frame around the code : L, single
  keepspaces=true,                 % keeps spaces in text, useful for keeping indentation of code (possibly needs columns=flexible)
  keywordstyle=\color{blue},       % keyword style
  language=C++,                 % the language of the code
  %morekeywords={*,...},            % if you want to add more keywords to the
  % set
  %numbers=left,                    % where to put the line-numbers; possible
  % values are (none, left, right)
  %numbersep=5pt,                   % how far the line-numbers are from the code
%  numberstyle=\tiny\color{mygray}, % the style that is used for the
  % line-numbers
  rulecolor=\color{black},         % if not set, the frame-color may be changed on line-breaks within not-black text (e.g. comments (green here))
  showspaces=false,                % show spaces everywhere adding particular underscores; it overrides 'showstringspaces'
  showstringspaces=false,          % underline spaces within strings only
  showtabs=false,                  % show tabs within strings adding particular underscores
  stepnumber=2,                    % the step between two line-numbers. If it's 1, each line will be numbered
%  stringstyle=\color{mymauve},     % string literal style
  tabsize=2,                       % sets default tabsize to 2 spaces
  title=\lstname                   % show the filename of files included with \lstinputlisting; also try caption instead of title
}

\lstset{aboveskip=0pt,belowskip=10pt}

\renewcommand{\path}[1]{\textbf{#1}}
\newcommand{\method}[1]{\textbf{#1}}
\newcommand{\var}[1]{\textit{#1}}
\newcommand{\class}[1]{\textbf{#1}}
\newcommand{\refcode}[1]{``\textit{#1}''}

\title{A tutorial for NA62Analysis: creating the VertexCDA and
Pi0Reconstruction analyzers.}
\author{Nicolas Lurkin}

\begin{document}

\maketitle
\footnotesize
This document will describe the process of creating the VertexCDA and
Pi0Reconstruction analyzers within the NA62Analysis framework. It is intended to
guide future analyzer authors by describing the complete procedure. Assumption
is done that the environment is already configured, NA62Analysis is installed
and the User directory has been created. The installation and configuration
procedure is described at
\url{http://sergiant.web.cern.ch/sergiant/NA62FW/html/analysis.html}.
Before starting, the framework environment should be set by sourcing the
env.(c)sh file in the \path{scripts/} folder of the user directory. Every shell command
given in this document is to be executed from the top of the user directory.

\section{VertexCDA}

\newcommand{\p}[1]{\vec{p}_{#1}}
\newcommand{\ap}[1]{|\vec{p}_{#1}|^2}
\newcommand{\lv}{\vec{l}}
\newcommand{\lz}{\vec{l}_{0}}
\newcommand{\C}[1]{\vec{A}_{#1}}
\newcommand{\x}[1]{\vec{x}_{#1}}

The aim of this analyzer is to implement an algorithm that will compute the Kaon
decay vertex and make it available to further analyzers. Though different
methods are available for this purpose, the focus is given on a closest distance
of approach (CDA) algorithm. The algorithm implemented by this analyzer can be
used for the class of processes $K^\pm\to C^\pm X$ where $K^\pm$ is the
incoming charged Kaon measured in GigaTracker, $C^\pm$ is a charged particle
measured in the Spectrometer and $X$ is any kind and any number of additional particle. Even if the
tracks are supposed to originate from the same point they are in practice never intersecting
because of the finite measurement precision. As long as the tracks are not parallel the CDA will
find the unique point $\vec{V}=(V_x,V_y,V_z)$ where the distance between $\vec{V}$ and the first
track and the distance between $\vec{V}$ and the second track are minimums. For this point only,
the vector $\lv$ passing through $\vec{V}$ and joining both tracks is perpendicular to
them.

We define $\vec{x}_{1,2}$ being the track origins and $\p{1,2}$ their
momenta. Any point $\C{1,2}$ on track can be described by the parameter $s_{1,2}$ such as
\begin{equation}
	\C{1,2} = \x{1,2} + s_{1,2} \p{1,2}
\end{equation}

The vector joining $\C{1}$ and $\C{2}$ is then
\begin{eqnarray}
	\lv(s_1,s_2)=\C{1}-\C{2} &=& \x{1}+s_1\p{1} - \x{2} - s_2\p{2}\\
%	&=& \x{1}-\x{2} + s_1\p{1} - s_2\p{2}\\
	&=& \lz + s_1\p{1} - s_2\p{2}
\end{eqnarray}
with $\lz = \x{1}-\x{2}$. Using the property that $\lv$ is perpendicular to
$\p{1}$ and $\p{2}$ when its length is minimum gives the following
equations:

\begin{align}
	\lv(s_{1c},s_{2c})\cdot\p{1} &= \lz\cdot\p{1} + s_{1c}\ap{1} -
	s_{2c}(\p{1}\cdot\p{2}) &= 0
	\\
	\lv(s_{1c},s_{2c})\cdot\p{2} &= \lz\cdot\p{2} + s_{1c}(\p{2}\cdot\p{1}) -
	s_{2c}\ap{2} &= 0
\end{align}
They can be solved for $s_{1c}$ and $s_{2c}$:

\begin{align}
	s_{1c} &= \frac{(\p{1}\cdot\p{2})(\lz\cdot\p{2}) -
	(\lz\cdot\p{1})\ap{2}}{\ap{1}\ap{2} - (\p{1}\cdot\p{2})^2}\\
	s_{2c} &= \frac{\ap{1}(\lz\cdot\p{2}) -
	(\p{1}\cdot\p{2})(\lz\cdot\p{1})}{\ap{1}\ap{2} - (\p{1}\cdot\p{2})^2}
\end{align}

The vertex being in the middle of the $\lv(s_{1c},s_{2c})$ vector:
\begin{equation}
	\vec{V} = \C{1} + 0.5\lv(s_{1c},s_{2c})
\end{equation}

\subsection{VertexCDA analyzer implementation}
The first step is to create the skeleton of the new analyzer. This is
automatically done with the framework python script:
\begin{lstlisting}
NA62AnalysisBuilder.py new VertexCDA
\end{lstlisting}

The source code of the newly created analyzer can be found in
\path{Analyzers/include/VertexCDA.hh} and \path{Analyzers/src/VertexCDA.cc}. Every standard method
of the analyzer and each section of code will be described thereafter.

\subsubsection{Constructor}
As the GigaTracker and Spectrometer information needs to be accessed, it is necessary to start by 
requesting these ROOT TTrees:

\begin{lstlisting}
//Request TTree with name GigaTracker with elements of class TRecoGigaTrackerEvent
RequestTree("GigaTracker", new TRecoGigaTrackerEvent);
//Request TTree with name Spectrometer with elements of class TRecoSpectrometerEvent
RequestTree("Spectrometer", new TRecoSpectrometerEvent);
\end{lstlisting}
 
 Without forgetting to include the relevant header files at the beginning of the file
\begin{lstlisting}
#include "TRecoGigaTrackerEvent.hh"
#include "TRecoSpectrometerEvent.hh"
\end{lstlisting}

\subsubsection{InitHist}
In this method all the histograms that will be needed during the processing are created and
registered to the framework:

\begin{itemize}
  \item Histograms for plotting the vertex x,y,z positions
\begin{lstlisting}
//Booking 3 histograms to plot the reconstructed vertex x,y,z positions
//Book Histogram VertexX (name that will be used to access this histogram later) and 
//create it with new TH1I (see ROOT TH1I for the syntax).
BookHisto("VertexX", 
	new TH1I("VertexX", "Reconstructed vertex X position", 250, -250, 250));
BookHisto("VertexY", 
	new TH1I("VertexY", "Reconstructed vertex Y position", 150, -150, 150));
BookHisto("VertexZ", 
	new TH1I("VertexZ", "Reconstructed vertex Z position", 100, 0, 300000));
\end{lstlisting}
	\item Histograms to compare the reconstructed vertex with the true Monte Carlo vertex
\begin{lstlisting}
//Booking 3 histograms to plot the difference between the reconstructed vertex and the real vertex components
BookHisto("DiffVertexX", 
	new TH1I("DiffVertexX", "X difference between reco and real vertex", 200, -50, 50));
BookHisto("DiffVertexY", 
	new TH1I("DiffVertexY", "Y difference between reco and real vertex", 200, -50, 50));
BookHisto("DiffVertexZ", 
	new TH1I("DiffVertexZ", "Z difference between reco and real vertex", 200, -10000, 10000));

//Booking 3 2D histograms to plot the reconstructed vertex components vs. the real vertex components
BookHisto("VertexRecoRealX", 
	new TH2I("VertexRecoRealX", "Reconstructed vs. Real (X)", 250, -250, 250, 250, -250, 250));
BookHisto("VertexRecoRealY", 
	new TH2I("VertexRecoRealY", "Reconstructed vs. Real (Y)", 150, -150, 150, 150, -150, 150));
BookHisto("VertexRecoRealZ", 
	new TH2I("VertexRecoRealZ", "Reconstructed vs. Real (Z)", 200, 0, 300000, 200, 0, 300000));
\end{lstlisting}
	\item Histograms for the particle multiplicity in detectors
\begin{lstlisting}
//Booking histograms to plot the reconstructed candidates multiplicity in GigaTracker and Spectrometer
BookHisto("GTKMultiplicity", 
	new TH1I("GTKMultiplicity", "Multiplicity in GTK", 11, -0.5, 10.5));
BookHisto("StrawMultiplicity", 
	new TH1I("StrawMultiplicity", "Multiplicity in Straw", 11, -0.5, 10.5));
\end{lstlisting}
	\item A serie of histogram to plot the kaon decay profile in 20 z sections
\begin{lstlisting}
//Booking a serie of 20 histogram to plot the MC Kaon decay vertex X-Y profile in different sections
of the detector length 
for(int i=0; i<20; i++){
	BookHisto(TString("BeamXY") + (Long_t)i, 
		new TH2I(TString("BeamXY") + (Long_t)i,
				TString("BeamXY") + (Long_t)(100+i*5) + TString("->") + (Long_t)(100+(i+1)*5),
				100, -100, 100, 100, -100, 100));
}
\end{lstlisting}
\end{itemize}

Several counters to keep track of the number of events passing the selection are created as well.
The counters are grouped in an \class{EventFraction} table and the sample size for this table is
defined as the \refcode{Total\_Events} counter.

\begin{lstlisting}
BookCounter("Total_Events");
BookCounter("Good_GTK_Mult");
BookCounter("Good_Straw_Mult");

NewEventFraction("Selection");
AddCounterToEventFraction("Selection", "Total_Events");
AddCounterToEventFraction("Selection", "Good_GTK_Mult");
AddCounterToEventFraction("Selection", "Good_Straw_Mult");
DefineSampleSizeCounter("Selection", "Total_Events");
\end{lstlisting} 

\subsubsection{InitOutput}
This analyzer is meant to provide a vertex to other analyzers. The \var{fVertex} member object is
first declared in the header of the analyzer:

\begin{lstlisting}
protected:
	TVector3 fVertex;
\end{lstlisting}
Before declaring it to the framework under the name \refcode{Vertex} in the \method{InitOutput}
method. It should be noted that to avoid collisions between independent analyzers, this name is
automatically prepended with the name of the analyzer and a dot. In this case, to access this
object from another analyzer, one will have to request \refcode{VertexCDA.Vertex}

\begin{lstlisting}
RegisterOutput("Vertex", &fVertex);
\end{lstlisting}

\subsubsection{DefineMCSimple}
This method is specific to simulated events. A specific partial event signature of the type
$K^+\to\pi^+X$ is defined where $X$ can be any number (0 included) of any particle. When reading
simulated events, the list of simulated particles (\class{KineParts}) will be scanned and the
particles corresponding to the definition given here will be stored in a structure that will allow
to easily referencing them later.

\begin{lstlisting}
//We ask for the initial Kaon and store it's ID
int kID = fMCSimple->AddParticle(0, 321);
// We ask for the positive pion whose parent is the initial Kaon (kID as parent)
fMCSimple->AddParticle(kID, 211);
\end{lstlisting}

\subsubsection{Process}
This method is the main process loop. Every event will be processed by this method. First some
variable declaration:
\begin{lstlisting}
//Flags for rejected/accepted events
bool badEvent = false;
//Will contain the Kaon position on GigaTracker station 3 and it's reconstructed momentum
TVector3 KaonPosition, KaonMomentum;
//Will contain the charged pion position returned by Spectrometer and it's reconstructed momentum 
TVector3 PipPosition, PipMomentum;
\end{lstlisting}

This analyzer does not need simulated data but can do extra-work for consistency checks if
available. The use of simulated data is therefore not enforced but it is checked anyway and the
\var{withMC} flag is set accordingly. The \var{fMCSimpe.fStatus} flag can take 3 different values:
\begin{itemize}
  \item \var{kEmpty}: No Monte Carlo data have been found
  \item \var{kMissing}: Monte Carlo data have been found but the current event does not correspond
  to the signature given in \method{DefineMCSimple} 
  \item \var{kComplete}: Monte Carlo data have been found and the event corresponds to the signature
  given in \method{DefineMCSimple}. There could be additional particles in the event.
\end{itemize}
The extra-work can only be realized with a complete event and therefore only events with the value
\var{KComplete} are relevant.

\begin{lstlisting}
//Declare the flag
bool withMC = true;
//For the additional jobWe are only interested in complete 
if(fMCSimple.fStatus != MCSimple::kComplete) withMC = false;
\end{lstlisting}

Then the events retrieved from the GigaTracker and Spectrometer TTrees are retrieved
\begin{lstlisting}
TRecoGigaTrackerEvent *GTKEvent = (TRecoGigaTrackerEvent*)GetEvent("GigaTracker");
TRecoSpectrometerEvent *SpectrometerEvent = (TRecoSpectrometerEvent*)GetEvent("Spectrometer");
\end{lstlisting}

And the counter that keeps track of the total number of processed events is incremented
\begin{lstlisting}
IncrementCounter("Total_Events");
\end{lstlisting}

The next step is handling the GigaTracker information. The multiplicity histogram are filled,
and the Kaon variables are set only if 1 candidate has been reconstructed. The Kaon position is the
position on station 3 but as the coordinates are relative to the centre of the GigaTracker
detector, it has to be translated into the experiment frame. Only the z coordinate is impacted.
The counter keeping the count of events passing the GigaTracker selection is incremented as well.
The event is flagged as bad if 0 or more than 1 candidates are reconstructed.

\begin{lstlisting}
//Filling GigaTracker multiplicity histogram
FillHisto("GTKMultiplicity", GTKEvent->GetNCandidates());
if(GTKEvent->GetNCandidates()==1){
	//If only 1 reconstructed candidate, set KaonPosition to the position on station 3 (index 2)
	KaonPosition = ((TRecoGigaTrackerCandidate*)GTKEvent->GetCandidate(0))->GetPosition(2);
	//Move Z to the experiment frame. Z position of GigaTracker centre is 90932.5mm  
	KaonPosition.SetZ(KaonPosition.Z()+90932.5);
	//Set the Kaon momentum as measured by GigaTracker
	KaonMomentum = ((TRecoGigaTrackerCandidate*)GTKEvent->GetCandidate(0))->GetMomentum().Vect();
	//Increment the GigaTracker selection counter
	IncrementCounter("Good_GTK_Mult");
}
else badEvent = true;
\end{lstlisting}

The same processing is applied for the reconstructed Spectrometer events. The multiplicity 
histogram is filled then the position and momentum are set if only one candidate is reconstructed
and the counter for the Spectrometer selection is incremented. If no or more than one event is
found, the \var{badEvent} flag is raised.

\begin{lstlisting}
FillHisto("StrawMultiplicity", SpectrometerEvent->GetNCandidates());
if(SpectrometerEvent->GetNCandidates()==1){
	PipPosition = ((TRecoSpectrometerCandidate*)SpectrometerEvent->GetCandidate(0))->GetPosition();
	PipMomentum = ((TRecoSpectrometerCandidate*)SpectrometerEvent->GetCandidate(0))->GetMomentum().Vect();
	IncrementCounter("Good_Straw_Mult");
}
else badEvent = true;
\end{lstlisting}

Finally if the event is tagged as bad, the state of the output is set to \var{kOInvalid} without
further treatment to signal other analyzers that the value of this output should not be used. 

\begin{lstlisting}
if(badEvent){
	//Signal the state of the Vertex output as kOInvalid if the event did not pass the selection
	SetOutputState("Vertex", kOInvalid);
}
\end{lstlisting}

If the event passed the selection, the vertex is finally computed with the CDA algorithm (whose
implementation is described in section \ref{GetIntersection}), the state of the output is set to
\var{kOValid} to signal other analyzers that they can use it, and the three Vertex histograms are
filled.

\begin{lstlisting}
else{
	//Get the vertex from the CDA algorithm
	fVertex = GetIntersection(KaonPosition, KaonMomentum, PipPosition, PipMomentum);
	\\Signal the state of the Vertex Output as kOValid
	SetOutputState("Vertex", kOValid);
	
	//Fill the histograms with the vertex components
	FillHisto("VertexX", fVertex.X());
	FillHisto("VertexY", fVertex.Y());
	FillHisto("VertexZ", fVertex.Z());
\end{lstlisting}

To finish the event processing, and if Monte Carlo are available, the reconstructed vertex is
compared to the real one component by component and the results are inserted in 1D and 2D
histograms. The vertex X-Y profile along the experiment is also plotted in 20 z sections starting
at z=100m.
\begin{lstlisting}
	if(withMC){
		FillHisto("DiffVertexX", fVertex.X()-fMCSimple["pi+"][0]->GetProdPos().X());
		FillHisto("DiffVertexY", fVertex.Y()-fMCSimple["pi+"][0]->GetProdPos().Y());
		FillHisto("DiffVertexZ", fVertex.Z()-fMCSimple["pi+"][0]->GetProdPos().Z());
		FillHisto("VertexRecoRealX", fVertex.X(), fMCSimple["pi+"][0]->GetProdPos().X());
		FillHisto("VertexRecoRealY", fVertex.Y(), fMCSimple["pi+"][0]->GetProdPos().Y());
		FillHisto("VertexRecoRealZ", fVertex.Z(), fMCSimple["pi+"][0]->GetProdPos().Z());
		int cat = ((fMCSimple["pi+"][0]->GetProdPos().Z()/1000)-100)/5;
		FillHisto(TString("BeamXY") + (Long_t)cat, fMCSimple["pi+"][0]->GetProdPos().X(), fMCSimple["pi+"][0]->GetProdPos().Y());
	}
}
\end{lstlisting}

\subsubsection{PostProcess}
This method is meant to destroy temporary objects that have been allocated during \method{Process}
and that shall be destroyed only after all the analyzers finished processing the event. This is
typically the case for memory allocated for the output of the analyzer. In the specific case of this
analyzer, no such memory has been allocated and this method will stay empty.

\subsubsection{ExportPlot}
All the histograms previously booked with \method{BookHisto} are saved in the output ROOT file with

\begin{lstlisting}
SaveAllPlots();
\end{lstlisting} 

\subsubsection{DrawPlot}
Similarly if the analysis is running in graphical mode, all the histograms previously booked with
\method{BookHisto} should be displayed on screen:

\begin{lstlisting}
DrawAllPlots();
\end{lstlisting} 

\subsubsection{GetIntersection}\label{GetIntersection}
This is the implementation of the CDA algorithm described earlier in this document. As input it
receive two 3-vectors \var{x1} and \var{x2} being respectively a point belonging to the first track
and a point belonging to the second track, and two 3-vector \var{p1} and \var{p2} representing the
directions of the tracks.

The \var{l0} vector is determined from the positions: 
\begin{lstlisting}
TVector3 l0 = x1-x2;
\end{lstlisting}
Then the different cross-products and the $s_{1,2}$ parameters are computed:
\begin{lstlisting}
double a = p1.Mag2();
double b = p1*p2;
double c = p2.Mag2();
double d = p1*d0;
double e = p2*d0;
double s1 = (b*e-c*d)/(a*c-b*b);
double s2 = (a*e-b*d)/(a*c-b*b);
\end{lstlisting}
The vector linking the closest points ($\lv(s_1, s_2)$ in the algorithm) is \var{vdist} and the
point $\vec{V}$ being in the middle of this vector is returned:
\begin{lstlisting}
TVector3 vdist = l0 + (s1*p1 - s2*p2);

return x1 + s1*p1 - 0.5*vdist;
\end{lstlisting}

\subsection{Validation}


\section{Pi0Reconstruction}

This analyzer is implementing the reconstruction of a $\pi^0$ candidate from two photon candidates
in the Liquid Krypton calorimeter. The eventual reconstructed candidate is then made available to
further analyzer as a \class{KinePart} object. 

xxxDescription of the algorithmxxx

\subsection{Pi0Reconstruction implementation}

The first step is to create the skeleton of the new analyzer. This is
automatically done with the framework python script:
\begin{lstlisting}
NA62AnalysisBuilder.py new Pi0Reconstruction
\end{lstlisting}

The source code of the newly created analyzer can be found in
\path{Examples/include/Pi0Reconstruction.hh} and \path{Examples/src/Pi0Reconstruction.cc}.
Every standard method of the analyzer and each section of code will be described thereafter.

\subsubsection{Constructor}
As this analyzer only needs information from the LKr, this is the only requested TTree. The
analyzer will run some acceptance checks as well and need access to the global instance of the
\class{DetectorAcceptance} object:

\begin{lstlisting}
RequestTree("LKr", new TRecoLKrEvent);
fDetectorAcceptanceInstance = GetDetectorAcceptanceInstance();
\end{lstlisting}

Without forgetting to include the header for the LKr reconstructed events

\begin{lstlisting}
#include "TRecoLKrEvent.hh"
\end{lstlisting}


\subsubsection{InitHist}
In this method all the histograms that will be needed during the processing are created and
registered to the framework:

\begin{itemize}
  \item Histograms for the photon candidates reconstructed energy 
\begin{lstlisting}
BookHisto("g1Energy", new TH1I("G1Energy", "Energy of g1", 100, 0, 75000));
BookHisto("g2Energy", new TH1I("G2Energy", "Energy of g2", 100, 0, 75000));
\end{lstlisting}
	\item 2D histograms to compare reconstructed photon candidates energy with the true Monte Carlo
	energy
\begin{lstlisting}
BookHisto("g1Reco", new TH2I("g1Reco", "g1 Reco vs. Real", 100, 0, 75000, 100, 0, 75000));
BookHisto("g2Reco", new TH2I("g2Reco", "g2 Reco vs. Real", 100, 0, 75000, 100, 0, 75000));
\end{lstlisting}
	\item 2D histograms to compare the reconstructed photon candidates positions on LKr with the
	extrapolated position of the true MC photons
\begin{lstlisting}
BookHisto("g1px", new TH2I("g1px", "g1 px Reco vs. Real", 200, 0, 2000, 200, 0, 2000));
BookHisto("g2px", new TH2I("g2px", "g2 px Reco vs. Real", 200, 0, 2000, 200, 0, 2000));
BookHisto("g1py", new TH2I("g1py", "g1 py Reco vs. Real", 200, 0, 2000, 200, 0, 2000));
BookHisto("g2py", new TH2I("g2py", "g2 py Reco vs. Real", 200, 0, 2000, 200, 0, 2000));
BookHisto("g1pz", new TH2I("g1pz", "g1 pz Reco vs. Real", 10, 240000, 250000, 10, 240000, 250000));
BookHisto("g2pz", new TH2I("g2pz", "g2 pz Reco vs. Real", 10, 240000, 250000, 10, 240000, 250000));
\end{lstlisting}
	\item Histograms for the reconstructed $\pi^0$ candidate properties
\begin{lstlisting}
BookHisto("pi0Energy", new TH1I("pi0Energy", "Energy of pi0", 100, 0, 75000));
BookHisto("pi0Mass", new TH1I("pi0Mass", "Reconstructed mass of pi0", 200, 0, 200));
BookHisto("pi0MCMass", new TH1I("pi0MCMass", "MC mass of pi0", 200, 0, 200));
\end{lstlisting}
	\item Histograms for LKr Monitoring
\begin{lstlisting}
BookHisto("clusterPosition", new TH2I("clusterPosition", "Cluster position on LKr", 500, -2000, 2000, 500, -2000, 2000));
BookHisto("photonsNbr", new TH1I("photonsNbr", "Photons number/event", 10, 0, 10));
BookHisto("energyCalib", new TGraph());
BookHisto("g1EnergyFraction", new TH1I("g1EnergyFraction", "Fraction between real energy and reco energy", 1000, 0, 100));
BookHisto("g2EnergyFraction", new TH1I("g2EnergyFraction", "Fraction between real energy and reco energy", 1000, 0, 100));
\end{lstlisting}
	\item Histograms for the pair selection algorithm
\begin{lstlisting}
BookHisto("gPairSelected", new TH1I("gPairSelected", "Pair of gamma selected for Pi0", 10, 0, 10));
\end{lstlisting}
	\item Histograms specific to Monte Carlo events
\begin{lstlisting}
BookHisto("g1FirstVol", new TH1I("g1FirstVol", "First touched volume for g1", 15, 0, 15));
BookHisto("g2FirstVol", new TH1I("g2FirstVol", "First touched volume for g2", 15, 0, 15));
BookHisto("pdgID", new TH1I("pdgID", "Non complete events : pdgID", 0, 0, 0));
\end{lstlisting}
\end{itemize}

\subsubsection{InitOutput}
This analyzer should provide further analyzers with a $\pi^0$ candidate if any is found. The output
object is first declared in the header:

\begin{lstlisting}
KinePart pi0;
\end{lstlisting}

And then registered in the framework under the name \refcode{pi0} in the \method{InitOutput}
method. It should be noted that to avoid collisions between independent analyzers, this name is
automatically prepended with the name of the analyzer and a dot. In this case, to access this
object from another analyzer, one will have to request \refcode{Pi0Reconstruction.pi0}
\begin{lstlisting}
RegisterOutput("pi0", &pi0);
\end{lstlisting}

\subsubsection{DefineMCSimple}
In this method, the specific event signature $K^+\to\pi^+X\pi^0\to\gamma\gamma X$ is defined where
$X$ can be any kind or any number (including 0) of additional particle. This will allow to do
extra-processing to assess the performances of the analyzer when running on on this kind of
simulated events.
\begin{lstlisting}
int kID = fMCSimple->AddParticle(0, 321); //ask for beam Kaon
fMCSimple->AddParticle(kID, 211); //ask for positive pion from initial kaon decay
int pi0ID = fMCSimple->AddParticle(kID, 111); //ask for positive pion from initial kaon decay
fMCSimple->AddParticle(pi0ID, 22); //ask for positive pion from initial kaon decay
fMCSimple->AddParticle(pi0ID, 22); //ask for positive pion from initial kaon decay
\end{lstlisting}


\subsubsection{Process}
All the necessary variables are first declared
\begin{lstlisting}
//Temporary variables
vector<KinePart*> photons;
KinePart *part;
TRecoLKrCandidate *lkrCand;
vector<pair<KinePart*, KinePart*> > candidates;
int i;

//Maps
map<double, int> photonOrder;
map<double, int>::reverse_iterator photonIt, photonIt2;
multimap<double, int> cID;
multimap<double, int>::iterator cIDIterator;

//For input from VertexCDA
TVector3 vertex, direction;
OutputState state;


//Variables for reconstruction
int LKrStartPos = 240413;
KinePart *g1,*g2;
double pi0Mass;
int iLead=-1;
int iTrail=-1;

//Variables for MC checks
double g1EnergyFrac = 0;
double g2EnergyFrac=0;
DetectorAcceptance::volume g1Vol, g2Vol;
bool g1LKr, g2LKr;
TVector3 g1reco, g2reco, g1real, g2real;
\end{lstlisting}

The event is retrieved from the LKr TTree and two calibration constants are declared for the LKr
cluster energy correction:
\begin{lstlisting}
//Get LKr event from TTree
TRecoLKrEvent *LKrEvent = (TRecoLKrEvent*)GetEvent("LKr");
//Calibration constants
double calibMult = 0.9744;
double calibConst = -366.5;
\end{lstlisting}

As stated previously some supplementary work can be done if running on specific Monte Carlo events.
The use of simulated events is not enforced but the \var{withMC} flag is set accordingly:
\begin{lstlisting}
bool withMC = true;
if(fMCSimple.fStatus == MCSimple::kMissing){
	for(int i=0; i<MCTruthEvent->GetNKineParts(); i++){
		FillHisto("pdgID", ((KinePart*)MCTruthEvent->GetKineParts()->At(i))->GetParticleName(), 1);
	}
	return;
}
if(fMCSimple.fStatus == MCSimple::kEmpty) withMC = false;
\end{lstlisting}

The vertex is required to reconstruct the photons momenta and is retrieved from the
\class{VertexCDA} analyzer and its validity if verified.
\begin{lstlisting}
//Get vertex from VertexCDA analyzer
vertex = *(TVector3*)GetOutput("VertexCDA.Vertex", state);
//Check we got the vertex. We cannot work without the vertex
if(state!=kOValid) return;
\end{lstlisting}

Then all the LKr cluster are transformed into photon candidates. The hypothesis is made that all
clusters are due to a photon and that they all originate from the vertex. In which case the
particle momenta is 
$\vec{p}=E_{cl}*\frac{(\vec{x}_{cl}-\vec{V})}{\left|\vec{x}_{cl}-\vec{V}\right|}$ where $E_{cl}$ is
the corrected cluster energy, $\vec{x}_{cl}$ is the position of the cluster on the LKr surface and
$\vec{V}$ is the vertex. The processing can only continue if at least two photons candidates are
found.
\begin{lstlisting}
//Loop over the LKr clusters and create a KinePart photon candidate
for(int i=0; i<LKrEvent->GetNCandidates(); i++){
	lkrCand = (TRecoLKrCandidate*)LKrEvent->GetCandidate(i);
	part = new KinePart();
	//Fill the cluster position histogram
	FillHisto("clusterPosition", lkrCand->GetClusterX()*10, lkrCand->GetClusterY()*10);
	//Set the LKr position (*10 to go from cm to mm)
	direction.SetXYZ(lkrCand->GetClusterX()*10, lkrCand->GetClusterY()*10, LKrStartPos);
	//Set the direction
	direction = direction - vertex;
	//Set the magnitude of the direction to the corrected cluster energy to form the momentum (photon hypothesis)
	direction.SetMag(lkrCand->GetClusterEnergy()*1000*calibMult + calibConst);
	//Assign the properties to the KinePart
	part->SetProdPos(TLorentzVector(vertex, 0));
	part->SetInitialMomentum(direction);
	part->SetInitialEnergy(lkrCand->GetClusterEnergy()*1000*calibMult + calibConst);
	//Push the candidate in the list
	photonOrder.insert(pair<double, int>(lkrCand->GetClusterEnergy()*1000*calibMult + calibConst, photons.size()));
	photons.push_back(part);
}
//Fill the photon multiplicity histogram
FillHisto("photonsNbr", photonOrder.size());

i=0;
//We need at leat 2 photons to reconstruct the pi0
if(photonOrder.size()>=2){
\end{lstlisting}

In most of the case more than two photon candidates are found. This is mainly due to the
contribution of the charged pion creating a cluster and the photons sometimes creating more than one
cluster. The ratio between the invariant mass of every possible photon candidate pair and the
$\pi^0$ mass is computed. The invariant mass is computed as $M = \sqrt{p_{g_1}^2 + p_{g_2}^2}$ where
$p_{g_{1,2}}$ are the quadri-momenta of the photon candidates. The pair whose ratio is closest to 1
(the pair with the invariant mass closest to the $\pi^0$ mass) is chosen.
\begin{lstlisting}
//Looping over possible photon pairs and computing invariant mass for each
photonIt2 = photonOrder.rbegin();
for(photonIt = photonOrder.rbegin(); photonIt != photonOrder.rend(); photonIt++){
	g1 = photons[photonIt->second];
	photonIt2 = photonIt;
	for(photonIt2++; photonIt2 != photonOrder.rend(); photonIt2++){
		g2 = photons[photonIt2->second];

		candidates.push_back(pair<KinePart*, KinePart*>(g1, g2));
		pi0Mass = sqrt(pow(g1->GetInitialEnergy() + g2->GetInitialEnergy(),2) - (g1->GetInitialMomentum() + g2->GetInitialMomentum()).Mag2());
		//insert the invariant mass divided by the pi0 mass in the map
		cID.insert(pair<double, int>(fabs((pi0Mass/134.9766)-1), i));
		i++;
	}
}
//Select the photon pair closest to the pi0 mass
FillHisto("gPairSelected", cID.begin()->second);
g1 = candidates[cID.begin()->second].first;
g2 = candidates[cID.begin()->second].second;
\end{lstlisting}

If Monte Carlo data are available the energy of the selected photon candidates are compared with the
simulated photons coming from the $\pi^0$ decay. Tests are also done to see wether they are in the
acceptance of the LKr or not. 
\begin{lstlisting}
//Are we working with MC?
if(withMC){

	//Select the most energetic photons coming from the pi0 decay
	if(fMCSimple["gamma"][0]->GetInitialEnergy() >= fMCSimple["gamma"][1]->GetInitialEnergy()) iLead = 0;
	else iLead = 1;
	iTrail = !iLead;

	//Compare the energy with the selected pair
	g1EnergyFrac = g1->GetInitialEnergy()/fMCSimple["gamma"][iLead]->GetInitialEnergy();
	g2EnergyFrac = g2->GetInitialEnergy()/fMCSimple["gamma"][iTrail]->GetInitialEnergy();

	//Are the 2 real photons in the LKr acceptance?
	fDetectorAcceptanceInstance->FillPath(fMCSimple["gamma"][iLead]->GetProdPos().Vect(), fMCSimple["gamma"][iLead]->GetInitialMomentum());
	g1Vol = fDetectorAcceptanceInstance->FirstTouchedDetector();
	g1LKr = fDetectorAcceptanceInstance->GetDetAcceptance(DetectorAcceptance::kLKr);
	fDetectorAcceptanceInstance->CleanDetPath();
	fDetectorAcceptanceInstance->FillPath(fMCSimple["gamma"][iTrail]->GetProdPos().Vect(), fMCSimple["gamma"][iTrail]->GetInitialMomentum());
	g2Vol = fDetectorAcceptanceInstance->FirstTouchedDetector();
	g2LKr = fDetectorAcceptanceInstance->GetDetAcceptance(DetectorAcceptance::kLKr);

	FillHisto("g1FirstVol", g1Vol);
	FillHisto("g2FirstVol", g2Vol);
}
\end{lstlisting}

If no Monte Carlo are available the photons are simply assumed to be in the LKr acceptance and the
rest of the processing is only done if the photons are inside LKr.
\begin{lstlisting}
else{
	g1Vol = DetectorAcceptance::kLKr;
	g2Vol = DetectorAcceptance::kLKr;
	g1LKr = true;
	g2LKr = true;
}

if( (g1Vol != DetectorAcceptance::kLAV && g1LKr == true)
	&& (g2Vol != DetectorAcceptance::kLAV && g2LKr == true)){
\end{lstlisting}

Again if simulated data are available, the positions of the simulated photons are extrapolated at
the LKr surface and compared with the selected candidates and the comparison histograms are filled.
\begin{lstlisting}
//Expected position on LKr
g1reco = propagate(g1->GetProdPos().Vect(), g1->GetInitialMomentum(), LKrStartPos);
g2reco = propagate(g2->GetProdPos().Vect(), g2->GetInitialMomentum(), LKrStartPos);

if(withMC){
	//Comparison reconstructed momenta and energies between reconstructed and real
	g1real = propagate(fMCSimple["gamma"][iLead]->GetProdPos().Vect(), fMCSimple["gamma"][iLead]->GetInitialMomentum(), LKrStartPos);
	g2real = propagate(fMCSimple["gamma"][iTrail]->GetProdPos().Vect(), fMCSimple["gamma"][iTrail]->GetInitialMomentum(), LKrStartPos);

	FillHisto("g1px", g1reco.X(), g1real.X());
	FillHisto("g2px", g2reco.X(), g2real.X());
	FillHisto("g1py", g1reco.Y(), g1real.Y());
	FillHisto("g2py", g2reco.Y(), g2real.Y());
	FillHisto("g1pz", g1reco.Z(), g1real.Z());
	FillHisto("g2pz", g2reco.Z(), g2real.Z());
	FillHisto("g1Reco", g1->GetInitialEnergy(), fMCSimple["gamma"][iLead]->GetInitialEnergy());
	FillHisto("g2Reco", g2->GetInitialEnergy(), fMCSimple["gamma"][iTrail]->GetInitialEnergy());
	//Dont do it if we are likely to have selected the wrong photon
	if(g1EnergyFrac>0.95) FillHisto("energyCalib", g1->GetInitialEnergy(), fMCSimple["gamma"][iLead]->GetInitialEnergy());
	if(g2EnergyFrac>0.95) FillHisto("energyCalib", g2->GetInitialEnergy(), fMCSimple["gamma"][iTrail]->GetInitialEnergy());
	FillHisto("g1EnergyFraction", g1EnergyFrac);
	FillHisto("g2EnergyFraction", g2EnergyFrac);
}

FillHisto("g1Energy", g1->GetInitialEnergy());
FillHisto("g2Energy", g2->GetInitialEnergy());
\end{lstlisting}

Finally the $\pi^0$ candidate is reconstructed. It's energy is the sum of the photons energies and
its momentum  is the sum of the photon momenta. The state of the output is set as \refcode{kOValid}
to signal further analyzer that the output can be used. The histograms corresponding to the $\pi^0$
candidate values are filled.
\begin{lstlisting}
//Reconstruct the pi0 candidate and create the KinePart
pi0.SetInitialEnergy(g1->GetInitialEnergy() + g2->GetInitialEnergy());
direction = g1->GetInitialMomentum();
direction.SetMag(g1->GetInitialEnergy());
g1->SetInitialMomentum(direction);
direction = g2->GetInitialMomentum();
direction.SetMag(g2->GetInitialEnergy());
g2->SetInitialMomentum(direction);
pi0.SetInitialMomentum(g1->GetInitialMomentum() + g2->GetInitialMomentum());

//Set the output state as valid (we have a candidate)
SetOutputState("pi0", kOValid);

//Fill the pi0 histograms
FillHisto("pi0Energy", pi0.GetInitialEnergy());
FillHisto("pi0Mass", sqrt(pow(pi0.GetInitialEnergy(),2) - pi0.GetInitialMomentum().Mag2()));
FillHisto("pi0MCMass", sqrt(pow(fMCSimple[22][iLead]->GetInitialEnergy() + fMCSimple[22][iTrail]->GetInitialEnergy(), 2) - (fMCSimple[22][iLead]->GetInitialMomentum() + fMCSimple[22][iTrail]->GetInitialMomentum()).Mag2()));
\end{lstlisting}


The last step is to release all the memory that has been allocated during the processing. This can
safely be done here as this memory is not part of the output. I memory had been allocated for the
output, the release would have been done in the \method{PostProcess} method.
\begin{lstlisting}
//Delete all the created KinePart
while(photons.size()>0){
	delete photons.back();
	photons.pop_back();
}
\end{lstlisting}

\subsubsection{PostProcess}
This method is meant to destroy temporary objects that have been allocated during \method{Process}
and that shall be destroyed only after all the analyzers finished processing the event. This is
typically the case for memory allocated for the output of the analyzer. In the specific case of this
analyzer, no such memory has been allocated and this method will stay empty.

\subsubsection{ExportPlot}
All the histograms previously booked with \method{BookHisto} are saved in the output ROOT file with

\begin{lstlisting}
SaveAllPlots();
\end{lstlisting} 

\subsubsection{DrawPlot}
Similarly if the analysis is running in graphical mode, all the histograms previously booked with
\method{BookHisto} should be displayed on screen:

\begin{lstlisting}
DrawAllPlots();
\end{lstlisting} 

\subsection{Validation}


\end{document}
