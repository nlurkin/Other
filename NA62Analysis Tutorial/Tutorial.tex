%%This is a very basic article template.
%%There is just one section and two subsections.
\documentclass{article}

\title{A tutorial for NA62Analysis: creating the VertexCDA and
Pi0Reconstruction analyzers.}

\begin{document}

\maketitle

This document will describe the process of creating the VertexCDA and
Pi0Reconstruction analyzers within the NA62Analysis framework. It is intended to
guide future analyzer authors by describing the complete procedure.

\section{VertexCDA}
The aim of this analyzer is to implement an algorithm that will compute the Kaon
decay vertex and make it available to further analyzers. Though different
methods are available for this purpose, the focus is given on a closest distance
of approach (CDA) algorithm. The algoritm implemented by this analyzer can be
used for the class of processes $K^\pm\to C^\pm+\ldots$ where $K^\pm$ is the
incoming charged kaon measured in GigaTracker and $C^\pm$ is a charged particle
measured in the Spectrometer. Even if the tracks are supposed to originate from
the same point they are in practice never intersecting because of the
finite measurement precision. The CDA will find the unique point
$v=(v_x,v_y,v_z)$ where $d(v,t_1)$ and $d(v,t_2)$, the distance between $v$ and
the first track $t_1$ and the distance between $v$ and the second track $t_2$
respectively, are minimum.

\subsection{Description of the CDA algorithm}
 Even if the GigaTracker and Spectrometer
tracks are supposed to be part of a single event The closest distance of approach algorithm computes


\end{document}
