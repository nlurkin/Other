%%This is a very basic article template.
%%There is just one section and two subsections.
\documentclass{article}

\usepackage{amsmath}
\usepackage{hyperref}
\usepackage{listings}
\usepackage{color}

\lstset{ %
  backgroundcolor=\color{white},   % choose the background color; you must add \usepackage{color} or \usepackage{xcolor}
  basicstyle=\tiny,        % the size of the fonts that are used for
  % the code
  breakatwhitespace=false,         % sets if automatic breaks should only happen at whitespace
  breaklines=true,                 % sets automatic line breaking
  %captionpos=b,                    % sets the caption-position to bottom
  commentstyle=\color{green},    % comment style
  %deletekeywords={...},            % if you want to delete keywords from the
  % given language
  %escapeinside={\%*}{*)},          % if you want to add LaTeX within your code
  %extendedchars=true,              % lets you use non-ASCII characters; for
  % 8-bits encodings only, does not work with UTF-8
  frame=L,                    % adds a frame around the code : L, single
  keepspaces=true,                 % keeps spaces in text, useful for keeping indentation of code (possibly needs columns=flexible)
  keywordstyle=\color{blue},       % keyword style
  language=C++,                 % the language of the code
  %morekeywords={*,...},            % if you want to add more keywords to the
  % set
  %numbers=left,                    % where to put the line-numbers; possible
  % values are (none, left, right)
  %numbersep=5pt,                   % how far the line-numbers are from the code
%  numberstyle=\tiny\color{mygray}, % the style that is used for the
  % line-numbers
  rulecolor=\color{black},         % if not set, the frame-color may be changed on line-breaks within not-black text (e.g. comments (green here))
  showspaces=false,                % show spaces everywhere adding particular underscores; it overrides 'showstringspaces'
  showstringspaces=false,          % underline spaces within strings only
  showtabs=false,                  % show tabs within strings adding particular underscores
  stepnumber=2,                    % the step between two line-numbers. If it's 1, each line will be numbered
%  stringstyle=\color{mymauve},     % string literal style
  tabsize=2,                       % sets default tabsize to 2 spaces
  title=\lstname                   % show the filename of files included with \lstinputlisting; also try caption instead of title
}

\renewcommand{\path}[1]{\textbf{#1}}
\newcommand{\method}[1]{\textbf{#1}}
\newcommand{\var}[1]{\textit{#1}}
\newcommand{\class}[1]{\textbf{#1}}
\newcommand{\refcode}[1]{``\textit{#1}''}

\title{A tutorial for NA62Analysis: creating the VertexCDA and
Pi0Reconstruction analyzers.}

\begin{document}

\maketitle
\footnotesize
This document will describe the process of creating the VertexCDA and
Pi0Reconstruction analyzers within the NA62Analysis framework. It is intended to
guide future analyzer authors by describing the complete procedure. Assumption
is done that the environment is already configured, NA62Analysis is installed
and the User directory has been created. The installation and configuration
procedure is described at
\url{http://sergiant.web.cern.ch/sergiant/NA62FW/html/analysis.html}.
Before starting the framework environment should be set by sourcing the
env.(c)sh file in the \path{scripts/} folder of the user directory. Every shell command
given in this document is to be executed from the top of the user directory.

\section{VertexCDA}
The aim of this analyzer is to implement an algorithm that will compute the Kaon
decay vertex and make it available to further analyzers. Though different
methods are available for this purpose, the focus is given on a closest distance
of approach (CDA) algorithm. The algoritm implemented by this analyzer can be
used for the class of processes $K^\pm\to C^\pm X$ where $K^\pm$ is the
incoming charged kaon measured in GigaTracker, $C^\pm$ is a charged particle
measured in the Spectrometer and $X$ is any kind and any number of additional particule. Even if the
tracks are supposed to originate from the same point they are in practice never intersecting
because of the finite measurement precision. The CDA will find the unique point
$\vec{v}=(v_x,v_y,v_z)$ where the distance between $v$ and the first track and
the distance between $v$ and the second track are minimums. For this point only,
the line $l$ passing through $v$ and joining both tracks is perpendicular to them.

\newcommand{\p}[1]{\vec{p}_{#1}}
\newcommand{\ap}[1]{|\vec{p}_{#1}|^2}
\newcommand{\lv}{\vec{l}}
\newcommand{\lz}{\vec{l}_{0}}
\newcommand{\C}[1]{\vec{A}_{#1}}
\newcommand{\x}[1]{\vec{x}_{#1}}

We define $\vec{x}_{1,2}$ being the track origin and $\p{1,2}$ it's
momentum. Any point $\C{1,2}$ on track can be described by a
single parameter $s_{1,2}$.
\begin{equation}
	\C{1,2} = \x{1,2} + s_{1,2} \p{1,2}
\end{equation}

The vector joining $\C{1}$ and $\C{2}$ is then
\begin{eqnarray}
	\lv(s_1,s_2)=\C{1}-\C{2} &=& \x{1}+s_1\p{1} - \x{2} - s_2\p{2}\\
	&=& \x{1}-\x{2} + s_1\p{1} - s_2\p{2}\\
	&=& \lz + s_1\p{1} - s_2\p{2}
\end{eqnarray}
with $\lz = \x{1}-\x{2}$. Using the property that $\lv$ is perpendicular to
$\p{1}$ and $\p{2}$ when it's length is minimum gives the following
equations:

\begin{align}
	\lv(s_{1c},s_{2c})\cdot\p{1} &= \lz\cdot\p{1} + s_{1c}\ap{1} -
	s_{2c}(\p{1}\cdot\p{2}) &= 0
	\\
	\lv(s_{1c},s_{2c})\cdot\p{2} &= \lz\cdot\p{2} + s_{1c}(\p{2}\cdot\p{1}) -
	s_{2c}\ap{2} &= 0
\end{align}
That we solve for $s_{1c}$ and $s_{2c}$:

\begin{align}
	s_{1c} &= \frac{(\p{1}\cdot\p{2})(\lz\cdot\p{2}) -
	(\lz\cdot\p{1})\ap{2}}{\ap{1}\ap{2} - (\p{1}\cdot\p{2})^2}\\
	s_{2c} &= \frac{\ap{1}(\lz\cdot\p{2}) -
	(\p{1}\cdot\p{2})(\lz\cdot\p{1})}{\ap{1}\ap{2} - (\p{1}\cdot\p{2})^2}
\end{align}

The vertex being in the middle of the $\lv(s_{1c},s_{2c})$ vector:
\begin{equation}
	v = \C{1} + 0.5\lv(s_{1c},s_{2c})
\end{equation}

\subsection{VertexCDA analyzer implementation}
The first step is to create the skeleton of the new analyzer. This is
automatically done with the framework python script:
\begin{lstlisting}
NA62AnalysisBuilder.py new VertexCDA
\end{lstlisting}

We can open the source code of the newly created analyzer in
\path{Analyzers/include/VertexCDA.hh} and \path{Analyzers/src/VetexCDA.cc}, go through
every standard method of the analyzer and describe each section of code

\subsubsection{Constructor}
As we will need to access the GigaTracker and Spectrometer information we first
start by requesting the access to these ROOT TTrees:

\begin{lstlisting}
//Request TTree with name GigaTracker with elements of class TRecoGigaTrackerEvent
RequestTree("GigaTracker", new TRecoGigaTrackerEvent);
//Request TTree with name Spectrometer with elements of class TRecoSpectrometerEvent
RequestTree("Spectrometer", new TRecoSpectrometerEvent);
\end{lstlisting}
 
 Without forgetting to include the relevant header files
\begin{lstlisting}
#include "TRecoGigaTrackerEvent.hh"
#include "TRecoSpectrometerEvent.hh"
\end{lstlisting}

\subsubsection{InitHist}
In this method we will create and register all the histograms that we will need
during the processing.

\begin{lstlisting}
//Booking 3 histograms to plot the reconstructed vertex x,y,z positions
//Book Histogram VertexX (name that will be used to access this histogram later) and 
//create it with new TH1I (see ROOT TH1I for the syntax).
BookHisto("VertexX", 
	new TH1I("VertexX", "Reconstructed vertex X position", 250, -250, 250));
BookHisto("VertexY", 
	new TH1I("VertexY", "Reconstructed vertex Y position", 150, -150, 150));
BookHisto("VertexZ", 
	new TH1I("VertexZ", "Reconstructed vertex Z position", 100, 0, 300000));

//Booking 3 histograms to plot the difference between the reconstructed vertex and the real vertex components
BookHisto("DiffVertexX", 
	new TH1I("DiffVertexX", "X difference between reco and real vertex", 200, -50, 50));
BookHisto("DiffVertexY", 
	new TH1I("DiffVertexY", "Y difference between reco and real vertex", 200, -50, 50));
BookHisto("DiffVertexZ", 
	new TH1I("DiffVertexZ", "Z difference between reco and real vertex", 200, -10000, 10000));

//Booking 3 2D histograms to plot the reconstructed vertex components vs. the real vertex components
BookHisto("VertexRecoRealX", 
	new TH2I("VertexRecoRealX", "Reconstructed vs. Real (X)", 250, -250, 250, 250, -250, 250));
BookHisto("VertexRecoRealY", 
	new TH2I("VertexRecoRealY", "Reconstructed vs. Real (Y)", 150, -150, 150, 150, -150, 150));
BookHisto("VertexRecoRealZ", 
	new TH2I("VertexRecoRealZ", "Reconstructed vs. Real (Z)", 200, 0, 300000, 200, 0, 300000));

//Booking histograms to plot the reconstructed candidates multiplicity in GigaTracker and Spectrometer
BookHisto("GTKMultiplicity", 
	new TH1I("GTKMultiplicity", "Multiplicity in GTK", 11, -0.5, 10.5));
BookHisto("StrawMultiplicity", 
	new TH1I("StrawMultiplicity", "Multiplicity in Straw", 11, -0.5, 10.5));

//Booking a serie of 20 histogram to plot the MC kaon decay vertex X-Y profile in different sections
of the detector length 
for(int i=0; i<20; i++){
	BookHisto(TString("BeamXY") + (Long_t)i, 
		new TH2I(TString("BeamXY") + (Long_t)i,
				TString("BeamXY") + (Long_t)(100+i*5) + TString("->") + (Long_t)(100+(i+1)*5),
				100, -100, 100, 100, -100, 100));
}
\end{lstlisting}

We can also create several counters to keep track of the number of events passing the selection for
this analyzer. The counters are grouped in an \class{EventFraction} table and the sample size for
this table is defined as the \refcode{Total\_Events} counter.

\begin{lstlisting}
BookCounter("Total_Events");
BookCounter("Good_GTK_Mult");
BookCounter("Good_Straw_Mult");

NewEventFraction("Selection");
AddCounterToEventFraction("Selection", "Total_Events");
AddCounterToEventFraction("Selection", "Good_GTK_Mult");
AddCounterToEventFraction("Selection", "Good_Straw_Mult");
DefineSampleSizeCounter("Selection", "Total_Events");
\end{lstlisting} 

\subsubsection{InitOutput}
This analyzer is meant to provide a vertex to other analyzers. We first declare the \var{fVertex}
member object in the header of the analyzer:

\begin{lstlisting}
protected:
	TVector3 fVertex;
\end{lstlisting}
before declaring it to the framework under the name \refcode{Vertex} in the \method{InitOutput}
method. It should be noted that to avoid collisions between independant analyzers, this name is
automatically prepended with the name of the analyzer and a dot. In this case, to access this
object from another analyzer, one will have to request \refcode{VertexCDA.Vertex}

\begin{lstlisting}
RegisterOutput("Vertex", &fVertex);
\end{lstlisting}

\subsubsection{DefineMCSimple}
This method is specific to simulated events. We will define a specific partial event signature of
the type $K^+\to\pi^+X$ where $X$ can be any number (0 included) of any particle. When reading simulated
events, the list of simulated particles (\class{KineParts}) will be scanned and the particles
corresponding to the definition given here will be stored in a structure that will allow us to easily reference
them later.

\begin{lstlisting}
//We ask for the initial Kaon and store it's ID
int kID = fMCSimple->AddParticle(0, 321);
// We ask for the positive pion whose parent is the initial Kaon (kID as parent)
fMCSimple->AddParticle(kID, 211);
\end{lstlisting}

\subsubsection{Process}
This method is the main process loop. Every event will be processed by this method. First we declare
some variables:
\begin{lstlisting}
//Flags for rejected/accepted events
bool badEvent = false;
//Will contain the kaon position on GigaTracker station 3 and it's reconstructed momentum
TVector3 KaonPosition, KaonMomentum;
//Will contain the charged pion position returned by Spectrometer and it's reconstructed momentum 
TVector3 PipPosition, PipMomentum;
\end{lstlisting}

This analyzer does not need simulated data but can do extra-work for consistency checks if
available. We therefore don't enforce the use of simulated data but we simply check for them and
set the \var{withMC} flag accordingly. The \var{fMCSimpe.fStatus} flag can take 3 different values:
\begin{itemize}
  \item \var{kEmpty}: No Monte Carlo data have been found
  \item \var{kMissing}: Monte Carlo data have been found but the current event does not correspond
  to the signature given in \method{DefineMCSimple} 
  \item \var{kComplete}: Monte Carlo data have been found and the event corresponds to the signature
  given in \method{DefineMCSimple}. There could be additional particles in the event.
\end{itemize}
The extra-work can only be realized with a complete event and we will therefore only check for the
value \var{KComplete} of the status flag.

\begin{lstlisting}
//Declare the flag
bool withMC = true;
//For the additional jobWe are only interested in complete 
if(fMCSimple.fStatus != MCSimple::kComplete) withMC = false;
\end{lstlisting}

Then we request the events retrieved from the GigaTracker and Spectrometer TTrees
\begin{lstlisting}
TRecoGigaTrackerEvent *GTKEvent = (TRecoGigaTrackerEvent*)GetEvent("GigaTracker");
TRecoSpectrometerEvent *SpectrometerEvent = (TRecoSpectrometerEvent*)GetEvent("Spectrometer");
\end{lstlisting}

We increment to counter that keeps track of the total number of processed events:
\begin{lstlisting}
IncrementCounter("Total_Events");
\end{lstlisting}

The next step is handling the GigaTracker information. We fill the multiplicity histogram
and if only 1 candidate has been reconstructed we fill the kaon variables. The kaon position is the
position on station 3 but as the coordinates are relative to the center of the GigaTracker detector,
we have to translate it into the experiment frame. Only the z coordinate is impacted. We also
increment the counter keeping the count of events passing the GigaTracker selection. If 0 or more
than 1 candidate is reconstructed, we flag the event as bad.

\begin{lstlisting}
//Filling GigaTracker multiplicity histogram
FillHisto("GTKMultiplicity", GTKEvent->GetNCandidates());
if(GTKEvent->GetNCandidates()==1){
	//If only 1 reconstructed candidate, set KaonPosition to the position on station 3 (index 2)
	KaonPosition = ((TRecoGigaTrackerCandidate*)GTKEvent->GetCandidate(0))->GetPosition(2);
	//Move Z to the experiment frame. Z position of GigaTracker centre is 90932.5mm  
	KaonPosition.SetZ(KaonPosition.Z()+90932.5);
	//Set the kaon momentum as measured by GigaTracker
	KaonMomentum = ((TRecoGigaTrackerCandidate*)GTKEvent->GetCandidate(0))->GetMomentum().Vect();
	//Increment the GigaTracker selection counter
	IncrementCounter("Good_GTK_Mult");
}
else badEvent = true;
\end{lstlisting}

The same processing is applied for the recostructed Spectrometer events. We fill the multiplicity
histogram then get the position and momentum if only one candidate is reconstructed and we increment
the counter for the Spectrometer selection. If no or more than one event is found, the
\var{badEvent} flag is raised.

\begin{lstlisting}
FillHisto("StrawMultiplicity", SpectrometerEvent->GetNCandidates());
if(SpectrometerEvent->GetNCandidates()==1){
	PipPosition = ((TRecoSpectrometerCandidate*)SpectrometerEvent->GetCandidate(0))->GetPosition();
	PipMomentum = ((TRecoSpectrometerCandidate*)SpectrometerEvent->GetCandidate(0))->GetMomentum().Vect();
	IncrementCounter("Good_Straw_Mult");
}
else badEvent = true;
\end{lstlisting}

Finally if the event is tagged as bad, we set the state of the output to \var{kOInvalid} without
further treatment to signal other analyzers that the value of this output should not be used. 

\begin{lstlisting}
if(badEvent){
	//Signal the state of the Vertex output as kOInvalid if the event did not pass the selection
	SetOutputState("Vertex", kOInvalid);
}
\end{lstlisting}

If the event passed the selection, we finally compute the vertex with the CDA algorithm (whose
implementation is described in section \ref{GetIntersection}), set the state of the output to
\var{kOValid} to signal other analyzers that they can use it and fill the three Vertex histograms.

\begin{lstlisting}
else{
	//Get the vertex from the CDA algorithm
	fVertex = GetIntersection(KaonPosition, KaonMomentum, PipPosition, PipMomentum);
	\\Signal the state of the Vertex Output as kOValid
	SetOutputState("Vertex", kOValid);
	
	//Fill the histograms with the vertex components
	FillHisto("VertexX", fVertex.X());
	FillHisto("VertexY", fVertex.Y());
	FillHisto("VertexZ", fVertex.Z());
\end{lstlisting}

To finish the event processing and if MonteCarlo are available, we compare the reconstructed vertex
to the real one component by component and insert the result in 1D and 2D histograms. We also plot
the vertex X-Y profile along the experiment in 20 Z sections starting at Z=100m. 
\begin{lstlisting}
	if(withMC){
		FillHisto("DiffVertexX", fVertex.X()-fMCSimple["pi+"][0]->GetProdPos().X());
		FillHisto("DiffVertexY", fVertex.Y()-fMCSimple["pi+"][0]->GetProdPos().Y());
		FillHisto("DiffVertexZ", fVertex.Z()-fMCSimple["pi+"][0]->GetProdPos().Z());
		FillHisto("VertexRecoRealX", fVertex.X(), fMCSimple["pi+"][0]->GetProdPos().X());
		FillHisto("VertexRecoRealY", fVertex.Y(), fMCSimple["pi+"][0]->GetProdPos().Y());
		FillHisto("VertexRecoRealZ", fVertex.Z(), fMCSimple["pi+"][0]->GetProdPos().Z());
		int cat = ((fMCSimple["pi+"][0]->GetProdPos().Z()/1000)-100)/5;
		FillHisto(TString("BeamXY") + (Long_t)cat, fMCSimple["pi+"][0]->GetProdPos().X(), fMCSimple["pi+"][0]->GetProdPos().Y());
	}
}
\end{lstlisting}

\subsubsection{GetIntersection}\label{GetIntersection}
\end{document}
