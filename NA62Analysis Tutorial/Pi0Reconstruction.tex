\section{Pi0Reconstruction}

This analyzer is implementing the reconstruction of a $\pi^0$ candidate from two photon candidates
in the Liquid Krypton calorimeter. The eventual reconstructed candidate is then made available to
further analyzer as a \class{KinePart} object. 

xxxDescription of the algorithmxxx

\subsection{Pi0Reconstruction implementation}

The first step is to create the skeleton of the new analyzer. This is
automatically done with the framework python script:
\begin{lstlisting}
NA62AnalysisBuilder.py new Pi0Reconstruction
\end{lstlisting}

The source code of the newly created analyzer can be found in
\path{Examples/include/Pi0Reconstruction.hh} and \path{Examples/src/Pi0Reconstruction.cc}.
Every standard method of the analyzer and each section of code will be described thereafter.

\subsubsection{Constructor}
As this analyzer only needs information from the LKr, this is the only requested TTree. The
analyzer will run some acceptance checks as well and need access to the global instance of the
\class{DetectorAcceptance} object:

\begin{lstlisting}
RequestTree("LKr", new TRecoLKrEvent);
fDetectorAcceptanceInstance = GetDetectorAcceptanceInstance();
\end{lstlisting}

Without forgetting to include the header for the LKr reconstructed events

\begin{lstlisting}
#include "TRecoLKrEvent.hh"
\end{lstlisting}


\subsubsection{InitHist}
In this method all the histograms that will be needed during the processing are created and
registered to the framework:

\begin{itemize}
  \item Histograms for the photon candidates reconstructed energy 
\begin{lstlisting}
BookHisto("g1Energy", new TH1I("G1Energy", "Energy of g1", 100, 0, 75000));
BookHisto("g2Energy", new TH1I("G2Energy", "Energy of g2", 100, 0, 75000));
\end{lstlisting}
	\item 2D histograms to compare reconstructed photon candidates energy with the true Monte Carlo
	energy
\begin{lstlisting}
BookHisto("g1Reco", new TH2I("g1Reco", "g1 Reco vs. Real", 100, 0, 75000, 100, 0, 75000));
BookHisto("g2Reco", new TH2I("g2Reco", "g2 Reco vs. Real", 100, 0, 75000, 100, 0, 75000));
\end{lstlisting}
	\item 2D histograms to compare reconstructed photon candidates momentum with the true Monte Carlo
	momentum
\begin{lstlisting}
BookHisto("g1px", new TH2I("g1px", "g1 px Reco vs. Real", 200, 0, 2000, 200, 0, 2000));
BookHisto("g2px", new TH2I("g2px", "g2 px Reco vs. Real", 200, 0, 2000, 200, 0, 2000));
BookHisto("g1py", new TH2I("g1py", "g1 py Reco vs. Real", 200, 0, 2000, 200, 0, 2000));
BookHisto("g2py", new TH2I("g2py", "g2 py Reco vs. Real", 200, 0, 2000, 200, 0, 2000));
BookHisto("g1pz", new TH2I("g1pz", "g1 pz Reco vs. Real", 10, 240000, 250000, 10, 240000, 250000));
BookHisto("g2pz", new TH2I("g2pz", "g2 pz Reco vs. Real", 10, 240000, 250000, 10, 240000, 250000));
\end{lstlisting}
	\item Histograms for the reconstructed $\pi^0$ candidate properties
\begin{lstlisting}
BookHisto("pi0Energy", new TH1I("pi0Energy", "Energy of pi0", 100, 0, 75000));
BookHisto("pi0Mass", new TH1I("pi0Mass", "Reconstructed mass of pi0", 200, 0, 200));
BookHisto("pi0MCMass", new TH1I("pi0MCMass", "MC mass of pi0", 200, 0, 200));
\end{lstlisting}
	\item Histograms for LKr Monitoring
\begin{lstlisting}
BookHisto("clusterPosition", new TH2I("clusterPosition", "Cluster position on LKr", 500, -2000, 2000, 500, -2000, 2000));
BookHisto("photonsNbr", new TH1I("photonsNbr", "Photons number/event", 10, 0, 10));
BookHisto("energyCalib", new TGraph());
BookHisto("g1EnergyFraction", new TH1I("g1EnergyFraction", "Fraction between real energy and reco energy", 1000, 0, 100));
BookHisto("g2EnergyFraction", new TH1I("g2EnergyFraction", "Fraction between real energy and reco energy", 1000, 0, 100));
\end{lstlisting}
	\item Histograms for the pair selection algorithm
\begin{lstlisting}
BookHisto("gPairSelected", new TH1I("gPairSelected", "Pair of gamma selected for Pi0", 10, 0, 10));
\end{lstlisting}
	\item Histograms specific to Monte Carlo events
\begin{lstlisting}
BookHisto("g1FirstVol", new TH1I("g1FirstVol", "First touched volume for g1", 15, 0, 15));
BookHisto("g2FirstVol", new TH1I("g2FirstVol", "First touched volume for g2", 15, 0, 15));
BookHisto("pdgID", new TH1I("pdgID", "Non complete events : pdgID", 0, 0, 0));
\end{lstlisting}
\end{itemize}

\subsubsection{InitOutput}
This analyzer should provide further analyzers with a $\pi^0$ candidate if any is found. The output
object is first declared in the header:

\begin{lstlisting}
KinePart pi0;
\end{lstlisting}

And then registered in the framework under the name \refcode{pi0} in the \method{InitOutput}
method. It should be noted that to avoid collisions between independent analyzers, this name is
automatically prepended with the name of the analyzer and a dot. In this case, to access this
object from another analyzer, one will have to request \refcode{Pi0Reconstruction.pi0}
\begin{lstlisting}
RegisterOutput("pi0", &pi0);
\end{lstlisting}

\subsubsection{DefineMCSimple}
In this method, the specific event signature $K^+\to\pi^+X\pi^0\to\gamma\gamma X$ is defined where
$X$ can be any kind or any number (including 0) of additional particle. This will allow to do
extra-processing to assess the performances of the analyzer when running on on this kind of
simulated events.
\begin{lstlisting}
int kID = fMCSimple->AddParticle(0, 321); //ask for beam Kaon
fMCSimple->AddParticle(kID, 211); //ask for positive pion from initial kaon decay
int pi0ID = fMCSimple->AddParticle(kID, 111); //ask for positive pion from initial kaon decay
fMCSimple->AddParticle(pi0ID, 22); //ask for positive pion from initial kaon decay
fMCSimple->AddParticle(pi0ID, 22); //ask for positive pion from initial kaon decay
\end{lstlisting}


\subsubsection{Process}

\subsubsection{ExportPlot}
All the histograms previously booked with \method{BookHisto} are saved in the output ROOT file with

\begin{lstlisting}
SaveAllPlots();
\end{lstlisting} 

\subsubsection{DrawPlot}
Similarly if the analysis is running in graphical mode, all the histograms previously booked with
\method{BookHisto} should be displayed on screen:

\begin{lstlisting}
DrawAllPlots();
\end{lstlisting} 

\subsection{Validation}
